\documentclass[a4paper]{article}
\usepackage[notheorems]{beamerarticle}

%% Pre-amble - commonly defined macros.

%% Packages
%\usepackage{xltxtra} 
%\setmainfont[Mapping=tex-text]{Linux Libertine O}
%\setromanfont[Mapping=tex-text]{Linux Libertine O}
%\setsansfont[Mapping=tex-text]{Linux Biolinum O}
%\setmonofont[Mapping=tex-text]{Linux Libertine Mono O}


\usepackage[textsize=tiny]{todonotes}
\DeclareSymbolFont{bbold}{U}{bbold}{m}{n}
\DeclareSymbolFontAlphabet{\mathbbold}{bbold}
\usepackage{authblk}

\usepackage{enumerate}
\usepackage{amsmath}
\usepackage{amsfonts}
\usepackage{amssymb}
\usepackage{amsbsy}
\usepackage{bbm}
\usepackage{isomath}
\usepackage{amsthm}
\usepackage{dsfont}
\usepackage{algorithm}
%\usepackage{algorithmic}
\usepackage{algpseudocode}
\usepackage{mathrsfs}
%\usepackage{paralist}
\usepackage{epsfig}
\usepackage{overpic}
\usepackage[caption=false]{subfig}
\usepackage{makeidx} 
\usepackage{natbib}
 \def\newblock{\hskip .11em plus .33em minus .07em} % important line

\usepackage{mathtools}
%\usepackage{comment}
\usepackage{pgf}
\usepackage{tikz}
\usepackage{pgfplots}
\usepackage{gnuplot-lua-tikz}
\usepackage{pgfplotstable}

\pgfplotsset{compat=newest}
\usetikzlibrary{external}
%\tikzexternalize
\usetikzlibrary{automata,topaths,shapes,arrows,fit,decorations.markings,mindmap}
\tikzstyle{utility}=[diamond,draw=black,fill=green!20,inner sep=0mm, minimum size=8mm]
\tikzstyle{select}=[rectangle,draw=black,fill=red!20,inner sep=0mm, minimum size=6mm]
\tikzstyle{hidden}=[dashed]
\tikzstyle{RV}=[circle,draw=black,inner sep=0mm,fill=blue!20,minimum size=6mm]


%\renewcommand\proofSymbol{\ensuremath{\blacksquare}}
%x\newcommand\qedsymbol{\ensuremath{\blacksquare}}



%\numberwithin{corollary}{section}
%\numberwithin{lemma}{section}
%\numberwithin{theorem}{section}
%\numberwithin{assumption}{section}
%\numberwithin{fact}{section}
%\numberwithin{definition}{section}
%\numberwithin{example}{section}
%\numberwithin{conjecture}{section}
%\numberwithin{remark}{section}
%\numberwithin{claim}{section}



\newcommand \E {\mathop{\mbox{\ensuremath{\mathbb{E}}}}\nolimits}
\newcommand \hE {\hat{\mathop{\mbox{\ensuremath{\mathbb{E}}}}\nolimits}}
\renewcommand \Pr {\mathop{\mbox{\ensuremath{\mathbb{P}}}}\nolimits}

\newcommand{\tuple}[1]{\left\langle #1\right\rangle }
\newcommand{\set}[1]{\left\{\, #1 \,\right\} }
\newcommand{\cset}[2]{\left\{\, #1 ~\middle|~ #2 \,\right\} }
\newcommand{\meas}[1]{\mu\left(#1\right)}
\newcommand{\indep}{\mathrel{\text{\scalebox{1.07}{$\perp\mkern-10mu\perp$}}}}

%% Special characters
%\newcommand\Simplex {{\mathbbold{\Delta}}}
\newcommand\Simplex {{\mathcal P}}
\newcommand\Reals {{\mathds{R}}}
\newcommand\Naturals {{\mathds{N}}} 


\newcommand \then{\Rightarrow}
\newcommand \defn {\mathrel{\triangleq}}

\newcommand \sDP {\stackrel{\text{\tiny{*}}}{\approx}}

%%\newcommand \defn {\triangleq}
%%\newcommand \defn {\equiv}
%%\newcommand \defn {\coloneq}
%%\newcommand \defn {\stackrel{\text{\tiny def}}{=}}
%%\newcommand \defn {\stackrel{\text{def}}{\hbox{\equalsfill}}}


\newcommand \argmax{\mathop{\rm arg\,max}}
\newcommand \argmin{\mathop{\rm arg\,min}}
\newcommand \dtan{\mathop{\rm dtan}}
\newcommand \sgn{\mathop{\rm sgn}}
\newcommand \trace{\mathop{\rm trace}}
\newcommand \conv{\mathop{\rm conv}}

\newcommand \onenorm[1]{\left\|#1\right\|_1}
\newcommand \pnorm[2]{\left\|#1\right\|_{#2}}
\newcommand \inftynorm[1]{\left\right\|#1\|_\infty}
\newcommand \norm[1]{\left\|#1\right\|}
\newcommand \trans[1]{#1^{\top}}

\newcommand \grad {\nabla}
\newcommand \dgrad {\vec{\nabla}}
\newcommand \dd{\,\mathrm{d}}

\DeclareMathAlphabet{\mathpzc}{OT1}{pzc}{m}{it}

\newcommand \Gaussian {\mathop{\mathpzc{N}}\nolimits}
\newcommand \Poisson {\mathop{\mathpzc{Poisson}}\nolimits}
\newcommand \Multinomial {\mathop{\mathpzc{Multinomial}}\nolimits}
\newcommand \Dirichlet {\mathop{\mathpzc{Dirichlet}}\nolimits}
\newcommand \Bernoulli {\mathop{\mathpzc{Bernoulli}}\nolimits}
\newcommand \Beta {\mathop{\mathpzc{Beta}}\nolimits}
\newcommand \Laplace {\mathop{\mathpzc{Lap}}\nolimits}
\newcommand \GammaDist {\mathop{\mathpzc{Gamma}}\nolimits}
\newcommand \Softmax{\mathop{\mathpzc{Softmax}}\nolimits}
\newcommand \Exp{\mathop{\mathpzc{Exp}}\nolimits}
\newcommand \Wishart{\mathop{\mathpzc{W}}\nolimits}
\newcommand \Uniform{\mathop{\mathpzc{Unif}}\nolimits}



%\newcommand \note[1] {{\color{red} \texttt{[#1]}}}
%\newcommand \note[2][red] {{\color{#1} \texttt{[#2]}}}

\renewcommand{\O}[1]{\mathcal{O}(#1)}
\newcommand{\aO}[1]{\tilde{\mathcal{O}}(#1)}

\DeclarePairedDelimiter{\abs}{\lvert}{\rvert}
\DeclarePairedDelimiter{\ceil}{\lceil}{\rceil}
\DeclarePairedDelimiter\floor{\lfloor}{\rfloor}




%%% macros to make things smalller
% For comparison, the existing overlap macros:
% \def\llap#1{\hbox to 0pt{\hss#1}}
% \def\rlap#1{\hbox to 0pt{#1\hss}}
\def\clap#1{\hbox to 0pt{\hss#1\hss}}
\def\mathllap{\mathpalette\mathllapinternal}
\def\mathrlap{\mathpalette\mathrlapinternal}
\def\mathclap{\mathpalette\mathclapinternal}
\def\mathllapinternal#1#2{%
           \llap{$\mathsurround=0pt#1{#2}$}}
\def\mathrlapinternal#1#2{%
           \rlap{$\mathsurround=0pt#1{#2}$}}
\def\mathclapinternal#1#2{%
           \clap{$\mathsurround=0pt#1{#2}$}}
           
 
%\newcommand \xdist[2] {\rho(#1, #2)}
\newcommand \xdist[2] {d(#1, #2)}
%\newcommand \xdist[2] {\rho\left(#1, #2\right)}
%\newcommand \pdist[2] {\Delta\left(#1, #2\right)}
\newcommand \pdist[2]{D\left(#1, #2\right)}
\newcommand \KL[2] {d\left(#1 \middle\| #2\right)}
\newcommand \beg {\beta}
\newcommand \family {\mathscr{P}}

\newcommand {\CS} {\mathcal{S}}
\newcommand {\CX} {\mathcal{X}}
\newcommand {\CY} {\mathcal{Y}}
\newcommand {\CZ} {\mathcal{Z}}
\newcommand {\CA} {\mathcal{A}}
\newcommand {\CB} {\mathcal{B}}

\newcommand{\narms}{K}


\newcommand {\sucht} {\mathop{\rm s.t.}}
\newcommand \fair {f}
\newcommand \util {u}
\newcommand \val {V}
\newcommand \rew {\rho}

\newcommand \Param {\Theta}
\newcommand \param {\theta}
\newcommand \vparam {\vectorsym{\theta}}
\newcommand \bel {\beta}
\newcommand \Bel {\CB}

\newcommand \MDPs {\ensuremath{\Theta}}
\newcommand \mdp {\ensuremath{\theta}}
\newcommand \Pols {\ensuremath{\mathcal{P}}}
\newcommand \pbs {{\pol_\bel^*}}
\newcommand \belt {\bel_t{\xi_t}}
\newcommand \beltn {\bel_{t+1}}
\newcommand \beltp {\bel_{t-1}}
\newcommand \hyper {h}

\newcommand \act {a}
\newcommand \Act {\mathcal{A}}
\newcommand \out {y}
\newcommand \Out {\mathcal{Y}}
\newcommand \obs {x}
\newcommand \sns {z}
\newcommand \Obs {\mathcal{X}}
\newcommand \Sns {\mathcal{Z}}
\newcommand \alg {\lambda}
\newcommand \pol {\pi}
\newcommand \pthom {\pol^{\textrm{Th}}}
\newcommand \pkthom {\pol_k^{\textrm{Th}}}
\newcommand \psd {\pol^{\textrm{SD}}}

\newcommand \Pbx[1] {\Pr_{\bel}{(#1 \mid x)}}
\newcommand \Pbxx[1] {\Pr_{\bel}{(#1 \mid x')}}
\newcommand \powerset[1]{2^{#1}}

\newcommand\ind[1]{\mathop{\mbox{\ensuremath{\mathbb{I}}}}\left\{#1\right\}}
\newcommand\Ind{\mbox{\bf{I}}}

\newcommand \gr[1] {\todo{GR: #1}}
\newcommand \cd[1] {\todo{CD: #1}}
\newcommand \yl[1] {\todo{YL: #1}}
%\newcommand \dp[1] {\todo{DP: #1}}


\pgfplotsset{
  marginal/.style={
    red,
    mark=o,
    line width=2pt
  },
  sample/.style={
    blue,
    mark=square,
    line width=2pt
  },
  bayes/.style={
    black,
    line width=2pt,
    dashed
  },
}

\newlength \fwidth

%%% macros to make things smalller
% For comparison, the existing overlap macros:
% \def\llap#1{\hbox to 0pt{\hss#1}}
% \def\rlap#1{\hbox to 0pt{#1\hss}}
\def\clap#1{\hbox to 0pt{\hss#1\hss}}
\def\mathllap{\mathpalette\mathllapinternal}
\def\mathrlap{\mathpalette\mathrlapinternal}
\def\mathclap{\mathpalette\mathclapinternal}
\def\mathllapinternal#1#2{%
           \llap{$\mathsurround=0pt#1{#2}$}}
\def\mathrlapinternal#1#2{%
           \rlap{$\mathsurround=0pt#1{#2}$}}
\def\mathclapinternal#1#2{%
           \clap{$\mathsurround=0pt#1{#2}$}}


         


%%% Local Variables:
%%% mode: latex
%%% End:



%%% Local Variables:
%%% mode: latex
%%% TeX-master: "presentation"
%%% End:


%\def\solution {1}


\title{Introductory assignment} 
\author{Christos Dimitrakakis}
\begin{document}
\maketitle

{\Large Student name:}
\vspace{0.5em}
\hrule
\vspace{1em}
The purpose of this assignment is to evaluate the background knowledge
of the students in the course. Please provide as precise and concise
answers as possible. This assignment is \textbf{not graded}, but note the \textbf{bonus point}.

Please write your answers \emph{directly on the sheet.}

\section{Logic}
In this section $A, B$ are events. $\neg A$ means that $A$ is not true. $A \wedge B$ means 'A and B' while $A
\vee B$ means A or B. $A \Rightarrow B$ means that A implies B (i.e. if A is true then B is true).
$A \Leftrightarrow B$ means that A and B are equivalent (if A is true then B is true and vice-versa).

\begin{exercise}
  If $A$ implies $B$, i.e. $A \Rightarrow B$ then, which statement holds true $\neg A, \neg B$, the negations of $A$ and $B$?
  \begin{enumerate}
  \item $\neg A \Rightarrow \neg B$, i.e. if A is not true then B is not true.
  \item $\neg B \Rightarrow \neg A$, i.e. if B is not true then A is not true.
  \end{enumerate}
\end{exercise}

\begin{exercise}
  Which of the following statements is correct?
  \begin{enumerate}
  \item $\neg (A \wedge B) = \neg (A \wedge B)$
  \item $\neg B \Rightarrow \neg A$, i.e. if B is not true then A is not true.
  \end{enumerate}
\end{exercise}


\section{Sets}
In this section $A, B$ are sets.

\begin{exercise}
  If $A, B$ are sets then
  $A \subset B$ means:
  \begin{enumerate}
  \item $A$ is a subset of $B$, i.e. $x \in A \Rightarrow x \in B$.
  \item $B$ is a subset of $A$, i.e. $x \in B \Rightarrow x \in A$.
  \end{enumerate}
\end{exercise}


\begin{exercise}
  If $A, B$ are sets then
  $A \cap B$ is:
  \begin{enumerate}
  \item The set of points that are in either $A$ or $B$, i.e. $\{x \in A \vee x \in B\}$
  \item The set of points that are both in $A$ and $B$, i.e. $\{x \in A \wedge x \in B\}$ 
  \end{enumerate}
\end{exercise}

\begin{exercise}
  If $A, B$ are sets then
  $A \cup B$ is:
  \begin{enumerate}
  \item The set of points that are in either $A$ or $B$, i.e. $\{x \in A \vee x \in B\}$
  \item The set of points that are both in $A$ and $B$, i.e. $\{x \in A \wedge x \in B\}$ 
  \end{enumerate}
\end{exercise}



\section{Probability theory}
In this section we consider probability as a measure, i.e. as a function from sets to $[0,1]$. All events are subsets of the universal set $\Omega$, so that $P(\Omega) = 1$, $P(\emptyset) = 0$.
\begin{exercise}
  If $A, B$ are mutually exclusive events i.e. $A \cap B = \emptyset$,  then 
  \[
  P(A \cup B) =
  \ifdefined \solution
  P(A) + P(B)
  \else
  \ldots
  \fi
  \]
\end{exercise}

\begin{exercise}[Conditional probability]
  If $A, B$ are two events, with $P(B) > 0$, then conditional probability is defined as
  \[
  P(A \mid B) \defn 
  \ifdefined \solution
  \frac{P(A \cap B)}{P(B)}
  \else
  \ldots
  \fi
  \]
\end{exercise}


\section{Random variables and statistics}


\begin{exercise}
  A real-valued random variable $x$ is simply a function
  $x : \Omega \to \Reals$, with $x(\omega)$ being random because
  $\omega \in \Omega$ is random.  Write the definition of the
  expectation of $x$, when $P(\omega)$ is the probability of $\omega$ and
  $x(\omega)$ is the value of $x$ for a given $\omega$:
  \[
  \E(x) = 
  \ifdefined \solution
  \sum_{\omega \in \Omega} x(\omega) P(\omega) 
  \else
  \ldots
  \fi
  \]
\end{exercise}



\section{Linear algebra}

\begin{exercise}
  If $\bx = x_1, \ldots, x_n$, $\by = y_1, \ldots, y_n$ are two column vectors in $\Reals^n$, what is their inner product:
  \[
  \bx \cdot \by = \bx^\top \by = 
  \ifdefined \solution
  \sum_{i=1}^n x_i y_i
  \fi
  \]
\end{exercise}


\section{Calculus}

\begin{exercise}
  If $f : \Reals \to \Reals$ is a real-valued twice-differentiable function, what are \emph{sufficient} conditions for $x_0$ to be a \emph{local maximum} of the function, i.e. there exists $\epsilon > 0$ so that $f(x_0) \geq f(x)$ for all $x : |x - x_0| < \epsilon$?
\ifdefined\solution
If $d f(x_0) /dx  = 0$ then
$x_0$ is either a saddle point, a maximum or a minimum. If in addition $d^2 f(x_0) /dx^2 < 0$, then $x_0$ is a maximum.
\fi
\end{exercise}

\section{Dessert}
\begin{exercise}
  There are 10 exercises in the previous section, with one point for each correct answer. What do you guess is your score? Take time to think about your score!

  My score for exercises 1 to 10 is probably.... : 
\end{exercise}



\end{document}

%%% Local Variables:
%%% mode: latex
%%% TeX-master: t
%%% End:
